\chapter{Metodika práce}
V tejto kapitole bude popísaná metodika práce skladajúca sa z nasledujúcich častí:
\label{kap:metodika}

\begin{enumerate}
	\item  teoretické východiská
	\item  popis modelov
	\item  spôsob získavania údajov a ich zdroje
	\item  deskriptívne štatistiky
\end{enumerate}

\newpage

\section{Súvisiace práce}

\subsection{Analýza spokojnosti študentov v Španielsku}
\qquad V roku 2008 bol v Španielsku publikovaný článok, týkajúci sa spokojnosti študentov vysokých škôl.\cite{spain} Dáta, ktoré boli použité, sa získavali pomocou dotazníku, ktorý bol dávaný končiacim študentom 1. vysokoškolského stupňa v rokoch 2001 až 2004. Autori okrem spokojnosti zisťovali, či študenti v danej dobe boli nezamestnaní, pracovali na čiatoční úväzok, alebo pracovali na trvalý pracovný pomer. Hlavným cieľom prieskumu bolo porovnanie vplyvu zamestnaneckého štatútu na spokojnsť študentov s ich štúdiom na vysokej škole. 

Ako už bolo spomenuté, dáta sa zbierali priebežne v rokoch 2001 až 2004. Ako cieľová skupina boli vybraní absolventi a študenti bakalárskeho programu v odbore informatika na verenej univerzite v Barcelone. Do prieskumu boli zahrnutí len tí študenti, ktorí už zložili skúšky a mali uhradené všetky úradné poplatky. Celkovo tak bolo do prieskumu zapojených 116 študentov. V dotazníku bolo študentom položených niekoľko otázok a ich úlohou bolo ku každej otázke priradiť hodnotenie od 0 po 10. Jednotlivé kategórie otázok, na ktoré študenti odpovedali boli celková spokojnosť so štúdijným programom, hodnotenie teoretických prednášok, spokojnosť a kvalitou a kvantitou praktických cvičení, hodnotenie priestorov fakulty, hodnotenie, či ich počas štúdia fakulta dostatočne pripravuje na pohodlný prechod na pracovný trh, dostupnosť knižničných služieb a spokojnosť s laboratórnym vybavením. Okrem samotného hodnotenia študentov si autori zároveň zaznamenali o študentoch aj niekoľko všeobecných informácií. Jednalo sa o štúdijný program, na ktorom boli študenti zapísaní, kde boli v odbore informatika ponúkané na danej fakulte manažment a informačné systémy, ďalšou informáciou bolo, akým spôsobom boli prijatí, kde boli tak isto dva spôsoby, buď boli prijatí na základe prospechu z odborných predmetov na strednej škole, alebo absolvovali všeobecný test štúdijných predpokladov, ďalej následovali informácie o dĺžke ich štúdia, známka, ktorú obdržali na štátnych skúškach a obhajobách záverečných prác, informácia o prípadných súbežných štúdijách na inej vysokej škole a nakoniec pohlavie a vek. Najdôležitejšími informáciami, na ktoré sa autori článku boli, či študent pracuje na polovičný alebo trvalý pracovný pomer, počet rokov štúdia, počas ktorých študent zároveň pracoval a aj študoval, počet pracovných pozícií, na krorých študenti pracovali a o tom, či ich pracovná pozícia bolo príbuzná štúdijnému odboru, ktorí študovali.

Dôvodom, prečo boli vybraní práve študenti informatiky, bolo práve zameranie daného prieskumu na pracujúcich študentov a autori zo štatistík usúdili, že títo študenti majú počas štúdia dostatok pracovných príležitostí ešte pred tým, ako zložia štátne skúšky.

Výsledkom štúdie bolo zistenie, čo sa týka všeobecnej spokojnosti so štúdiom, tak spokojnejší boli študenti, ktorí získali lepšie známky, a ženy. Ďalším zistením bolo, že absolventi štúdijného programu informačné systémy boli spokojnejší ako abolventi manažmentu. Iným prekvapivým zistením bola tiež informácia o tom, že študenti, ktorí študovali dlhšie, boli so štúdijom výrazne spokojnejší ako tí, ktorí študovali kratšie. Autori predpokladajú, že príčinou toho, prečo sú študenti informačných systémov je, že zataiľ, čo študenti informačných systémov majú počas celého svojho štúdia len predmety, ktoré sa týkajú výhradne informatiky, tak študenti manažmentu majú okrem informatických predemtov aj iné, ktoré sa týkajú ekonomických tém.

Čo sa týka hlavného zamerania tejto štúdie ohľadom vplyvu zamestnanosti na spokojnosť so štúdijom, autori zistili, že študenti, ktorí viac pracovali na trvalý pracovný pomer, boli vo všeobecnosti menej spokojní ako tí, čo pracovali na polovičný pracovný úväzok. Autori sa domnievajú, že dôvodom bude to, že veľká väčšina z nich je zapísaná na externé štúdiom, čo má za príčinu, že sa menej stretávajú so svojimi kolegami, nie sú súčasťou študenstkého prostredia a neúčastnia sa spoločenských aktivít. Ďalším zistením, prečo boli externí študenti nespokojneší bolo, že očakávali od štúdia predmety, ktoré im najviac pomôžu do praxe, teda za ich nespokojnosť mohli najmä teoreticky zamerané premety. Keď sa autori pozreli na počet rokov, ktoré študenti odpracovali mal negatívny vplyv na ich pokojnosť so štúdiom, i keď sa ukázalo, že nezáleželo, ako dlho študenti pracovali, ale významný vplyv mal skôr fakt, že či daný študent niekedy pracoval. Čo sa týka zvyšných kategórií, na ktoré dávali študenti odpovede v dotazníku ukázalo sa, že ich priemerné hodnotenie sa pohybovalo okolo, hodnoty 7, avšak si všimli, že priemerná hodnota odpovede na otázku, či ich vysoká škola dostatočne pripravuje na vstup na pracovný trh, nadobúda hodnotu nižšiu ako 6. Z toho autori usúdili, že prioritou študentov je, aby ich vysoká škola pripravovala do ich budúceho zamestnania, čo vyplýva aj výsledkov, že extrení študenti, ktorí pracovali na trvalý pracovný pomer, boli významne nespokojnejší so štúdijom.  

Nakoniec spomenieme rozdiel medzi spomínaným článkom a našou témou bakalárskej práce. Ako už bolo spomenuté, v danej štúdie boli autori článku zameraní najmä na vplyv odpracovanej doby na spokojnosť so štúdiom. Avšak v našej práci sa sústredíme výlučne len na všeobecnú spokojnosť študentov so štúdijom na vysokej škole a vôbec sa nesústredíme na vplyv odpracovanej doby, keďže sme takéto informácie ani nemerali. Ďalším významným rozdielom je kvantita dát a ich rôznorodosť. Zatiaľ čo autori článku pozorovali len študentov z jednej fakulty a jedného odboru, kde ich počet bol 116, tak my disponujeme dátami o vyše 17000 študentoch zo všetkých verejných vysokých škôl z celého Slovenska, teda máme tam aj viacero odborov a programov. Poslednými 3 rozdielmi sú typy otázok, ktoré boli položené našim študentom, ďalej študenti v našom prieskume boli z rôznych ročníkov, teda nie len tí, ktorí už zložili štátne skúšky a náš prieskum sa uskutočnil len počas jedného akademického roka a nie počas viacerých. \cite{prieskum}